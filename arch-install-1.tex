\title{Arch Linuxのインストール}
\author{おがわ}
\begin{document}
  \begin{frame}
    \note {
      動画全体ので説明する項目概要\\
      用語について、随時説明を加えています。
    }
    \frametitle{Arch Linuxのインストール}
    \begin{enumerate}
      \item ハードウェア、ファーム(UEFI)の仕様
        \note [item] {
          OSのインストーラソフト(Arch linux)は、ファーム(UEFI)の仕様に従いハードウェアを認識します。仕様を理解しておくと、動作しない場合の対策を立てやすくなります。
        }
      \item OSのファイルシステム
        \note [item] {
          ストレージデバイス(ssd)のパーティション設定、各パーティションのファイルフォーマットの選定と設定をおこないます。
        }
      \item Linuxのインストールからネットワークの接続設定(wifi)
        \note [item] {
          インストーラソフトによるストレージデバイスへのLinuxカーネルと必要なソフトウェアのインストールを行います。
          ネットワーク接続設定のためのソフトウェアの選定、インストールを行います。
        }
    \end{enumerate} 
  \end{frame}
  \begin{frame}
    \frametitle{ハードウェア構成}
    \note{
      インストール対象のハードウェアは次のとおりです。
      裏面の蓋をあけて中身を確認していきます。
    }
    \newcommand{\selectspec}{\usebeamercolor[fg]{alerted text}}
    \begin{center}
      \begin{tabular}{l|l}
        \hline
        \temporal<2>{CPU}{\selectspec{CPU}}{CPU}%
          & \temporal<2>{AMD Ryzen7 8845HS}
            {\selectspec{AMD Ryzen7 8845HS}}
            {AMD Ryzen7 8845HS} \\ \hline
        \temporal<3>{メモリ}{\selectspec{メモリ}}{メモリ}%
          & \temporal<3>{64 GB}
            {\selectspec{64 GB}}{64 GB} \\  \hline
        \temporal<4>{SSD}{\selectspec{SSD}}{SSD}%
          & \temporal<4>{2 TB}{\selectspec{2 TB}}{2 TB} \\ \hline
        \temporal<5>{グラフィック}{\selectspec{グラフィック}}{グラフィック}%
          & \temporal<5>{Radeon 780M}
            {\selectspec{Radeon 780M}}
            {Radeon 780M} \\ \hline
      \end{tabular}
    \end{center}
    \begin{center}
      \includegraphics<1>[height=4cm, page=1]{img/hw-img.pdf}%
      \includegraphics<2,5>[height=4cm, page=13]{img/hw-img.pdf}%
      \includegraphics<3>[height=4cm, page=15]{img/hw-img.pdf}%
      \includegraphics<4>[height=4cm, page=14]{img/hw-img.pdf}%
    \end{center}
    \note[item]{cpuはAMD Ryzen7。本体の裏面の蓋を開けた状態の画像からは確認できません。} 
    \note[item]{memoy 32GBのメモリカード2枚で64GB}
    \note[item]{ストレージはssdで、2TBを設置}
    \note[item]{グラフィックはRadeon。こちらも画像からは確認できません。}
  \end{frame}
  \begin{frame}
    \frametitle{入出力ポート} 
    \note{
      入出力ポートについては、次のものが利用できます。
    }
    \newcommand{\selectspec}{\usebeamercolor[fg]{alerted text}}
    \begin{center}
      \begin{tabular}{l|l|l}
        \hline
        \temporal<1>{USB 3.2}{\selectspec{USB 3.2}}{USB 3.2}%
          & \temporal<1>{2}{\selectspec{2}}{2}%
          & \\ \hline
        \temporal<2>{USB 2.0}{\selectspec{USB 2.0}}{USB 2.0}%
          & \temporal<2>{2}{\selectspec{2}}{2} & \\ \hline
        \temporal<3>{HDMI}{\selectspec{HDMI}}{HDMI}%
          & \temporal<3>{2}{\selectspec{2}}{2} & \\ \hline
        \temporal<4>{DP}{\selectspec{DP}}{DP}%
          & \temporal<4>{2}{\selectspec{2}}{2} & \\ \hline
        \temporal<5>{シリアルポート}
            {\selectspec{シリアルポート}}{シリアルポート}%
          & \temporal<5>{1}{\selectspec{1}}{1}%
          & \temporal<5>{RS232、RS485}
            {\selectspec{RS232、RS485}}{RS232、RS485} \\ \hline
      \end{tabular}
      \note[item] {USB3.2のタイプAのポートが2つ。位置は前面}
      \note[item] {USB2.0のタイプAのポートが2つ。位置は背面}
      \note[item] {HDMIのポートが2つ。位置は背面}
      \note[item] {ディスプレイポート2.1が2つ。位置は背面}
      \note[item] {シリアルのポートが1つ。位置は背面。今すぐの利用は考えていません。}
    \end{center}
    \begin{center}
      \includegraphics<1>[height=4cm, page=3]{img/hw-img.pdf}%
      \includegraphics<2>[height=4cm, page=4]{img/hw-img.pdf}%
      \includegraphics<3>[height=4cm, page=5]{img/hw-img.pdf}%
      \includegraphics<4>[height=4cm, page=6]{img/hw-img.pdf}%
      \includegraphics<5>[height=4cm, page=7]{img/hw-img.pdf}%
    \end{center}
  \end{frame}
  \begin{frame}
    \frametitle{ネットワーク} 
    \note{
      ネットワーク関連は、表のようになっています。
    }
    \newcommand{\selectspec}{\usebeamercolor[fg]{alerted text}}
    \begin{center}    
      \begin{tabular}{l|l}
        \hline
        \temporal<1>{イーサネット}{\selectspec{イーサネット}}{イーサネット}%
        & \temporal<1>{2}{\selectspec{2}}{2} \\ \hline
        \temporal<2>{Wi-Fi}{\selectspec{Wi-Fi}}{Wi-Fi}%
        & \temporal<2>{1}{\selectspec{1}}{1}\\
        \hline
      \end{tabular}
      \note[item] {イーサネットポートが2つ。位置は背面}
      \note[item] {Wi-Fiが1つ。アンテナが背面に確認できます}
    \end{center}
    \begin{center}
      \includegraphics<1>[height=4cm, page=8]{img/hw-img.pdf}%
      \includegraphics<2>[height=4cm, page=9]{img/hw-img.pdf}%
    \end{center}
  \end{frame}
  \begin{frame}
    \frametitle{ファーム}
    \note{
      AMI社のファームで、UEFI仕様に従っています。
      起動ボタン押下直後にF2で設定が変更できます。
    }
    \begin{center}
    \includegraphics[height=5cm]{img/bios-display-1.jpg}
    \end{center}
  \end{frame}
\end{document}
% vi: se ts=2 sw=2 et:
